\documentclass[a4paper, 12pt]{article}

\input{/home/nick/latex-preambles/xelatex.tex}

\setmainfont{Minion Pro}

\newcommand{\imagesPath}{.}

\title{
	\textbf{Εργαστήριο Δικτύων Υπολογιστών} \\~\\
	Λύσεις παλιών θεμάτων
}

\author{Νίκος Παγώνας}
\date{}

\begin{document}

\maketitle

\section{}
	Θα διαγραφεί από τον πίνακα δρομολόγησης.
	
\section{}
	ifconfig em0 inet6 fd00:2001::1/64.

\section{}
	show ip bgp summary

\section{}
	\textcolor{red}{Δεν την προσθέτει στην RIB}

\section{}
	Ώστε τελικά όλοι να αποκτήσουν την ίδια γνώση για τις διαδρομές.

\section{}
	Default και Intra-area.

\section{}
	Established.

\section{}
	Επειδή σε όλες τις διεπαφές οι τοπικές στη ζεύξη (link-local) διευθύνσεις έχουν όλες το ίδιο
	πρόθεμα δικτύου (fe80::/64).

\section{}
	\textcolor{red}{Θα αγνοήσει τη νέα διαδρομή}

\section{}
	Διότι θα μπορούσε κάποιος που γνωρίζει την τοπολογία του δικτύου να θέσει επίτηδες την κατάλληλη τιμή ώστε να φαίνεται πως το πακέτο προέρχεται από δρομολογητή.

\section{}
	tcpdump -i em0 "tcp and dst 10.0.10.10 and src port 22"

\section{}
	Την απορρίπτει και δεν την γράφει στην RIB.

\section{}
	2-way.

\section{}
	show ip bgp neighbors 192.168.2.1 routes

\section{}
	Εάν δεν υπάρξει αλλαγή για 30 min σε κάποια κατάσταση ζεύξης, ο δρομολογητής που παρήγαγε
	το σχετικό LSA, αναπαράγει την ενημέρωση και την ξαναστέλνει.

\section{}
	neighbor 192.168.3.2 remote-as 65020

\section{}
	Τα διαβιβαστικά έχουν δύο ή περισσότερους δρομολογητές OSPF και τα πακέτα μπορούν να πηγάζουν ή διέρχονται από αυτά, ενώ τα δίκτυα απολήξεις έχουν ένα μόνο δρομολογητή OSPF και τα πακέτα είτε πηγάζουν είτε καταλήγουν σε αυτά.

\section{}
	show ip bgp neighbors 192.168.2.1 advertised-routes

\section{}
	Οι ABR περιοχής που περιέχει ένα ASBR

\section{}
	172.17.0.0/16, 172.18.0.0/17, 172.19.0.0/18

\section{}
	ORIGIN, AS\_PATH και NEXT\_HOP

\section{}
	show ip bgp

\section{}
	TCP @ 179

\section{}
	Επιλεγμένες

\section{}
	Το πακέτο πέρασε από δρομολογητή ο οποίος δεν είχε στον πίνακα δρομολόγησης εγγραφή σχετική με τον προορισμό, ή έχουμε την περίπτωση traceroute με ICMP μηνύματα.

\section{}
	RIP 120, Internal BGP 200, OSPF 110, External BGP 20

\section{}
	\textcolor{red}{...}

\section{}
	default-information originate

\section{}
	Intra-area, Inter-area και Default

\section{}
	True

\section{}
	network 192.168.2.0/24 area 2

\section{}
	Δεν λαμβάνεται απάντηση από τον προορισμό, οπότε θεωρείται ότι είναι ανενεργός.

\section{}
	\textcolor{red}{...}

\section{}
	route add default 172.17.17.4

\section{}
	τους γείτονές του και αυτόν από όπου έλαβε την ενημέρωση

\section{}
	ifconfig em0 192.168.2.1/24

\section{}
	τον τύπο πηγής ORIGIN

\section{}
	route -6 add default fd02:2002::1

\section{}
	δεν στέλνει BPDU

\section{}
	Διεπαφή 64, δίκτυο 64

\section{}
	Δεν υπάρχει εγγραφή σχετική με τον προορισμό στον πίνακα δρομολόγησης, ούτε προκαθορισμένη διαδρομή.

\section{}
	802.3

\section{}
	Η διαδικασία συγχρονισμού στην περίπτωση
	λειτουργίας πολλών δρομολογητών εντός ενός υποδικτύου τύπου εκπομπής θα ήταν μη αποδοτική. Ο επιλεγμένος δρομολογητής έχει τον ειδικό ρόλο της διάδοσης μέσω αυτού όλων των ενημερώσεων. Σε περίπτωση αστοχίας του αναλαμβάνει το έργο αυτό ο BDR.

\section{}
	show ip route bgp

\section{}
	tcpdump -i em0 "tcp port 179"

% 46
\section{}
	\begin{verbatim}
		R2:
	
		ASBR - yes
		backbone - yes
		ABR - yes
		BDR - no
		DR - yes
		internal - no  
	\end{verbatim}

% 47
\section{}
	\begin{verbatim}
		R4:
		
		192.0.2.48/28 ---
		192.0.2.16/28 C
		192.0.2.192/26 C
		192.0.2.128/26 --- 
		192.0.2.32/28 --- 
		0.0.0.0/0 --- 
		192.0.2.64/26 R
		192.0.2.0/28 C
	\end{verbatim}

% 48
\section{}
	\begin{verbatim}
		R3:
		
		192.0.2.48/28 ---
		192.0.2.32/28 C
		192.0.2.128/26 C
		192.0.2.0/28 R
		192.0.2.192/26 R 
		0.0.0.0/0 ---
		192.0.2.64/26 O 
		192.0.2.16/28 C
	\end{verbatim}

% 49 
\section{}
	\begin{verbatim}
		R2:
		
		0.0.0.0/0 B
		192.0.2.48/28 C 
		192.0.2.64/26 C
		192.0.2.32/28 C
		192.0.2.128/26 O
		192.0.2.16/28 ---
		192.0.2.0/28 O
		192.0.2.192/26 O 
	\end{verbatim}	

% 50
\section{}
	ip address 192.0.8.1/24

% 51
\section{}
	N/A
	

% 52
\section{}
	N/A
	
% 53
\section{}
	\begin{verbatim}
		R3:
	
		DR - no 
		ABR - yes
		ASBR - yes
		internal - no 
		BDR - yes
		backbone - yes
	\end{verbatim}

% 54
\section{}
	Router, Network, Summary
\end{document}